Esse git representa um trabalho de jogo que será entregue na diciplina de PDSII -\/ TW

Diretórios
\begin{DoxyItemize}
\item 
\item banco\+Dados/\+: Contém dados dos jogadores e das frases faladas pela GLa\+DOS
\item bin/\+: Armazena os executáveis gerados pela compilação.
\item coverage/\+: saidas dos testes executados e cobertura do codigo.
\item html/\+: Documentação do codigo em html.
\item include/\+: Contém arquivos de cabeçalho (.hpp) usados no projeto.
\item latex/\+: Documentação do codigo em latex.
\item obj/\+: Guarda arquivos objeto (.o ou .obj) criados durante a compilação.
\item src/\+: Onde estão os arquivos de código-\/fonte (.cpp).
\item tests/\+: Abriga arquivos de teste para validação do código.
\item wiki\+\_\+images/\+: Imagens dos cartões CRC usados no Wiki
\item .gitignore\+: Lista arquivos e pastas que o Git deve ignorar.
\item makefile\+: Contém instruções para automatizar a compilação do projeto.
\end{DoxyItemize}

O Trabalho
\begin{DoxyItemize}
\item O objetivo deste código é implementar o funcionamento de alguns jogos de tabuleiro. Para isso é necessário conhecer e implementar as regras dos jogos escolhidos, que foram\+:
\item Jogo da Velha -\/ \href{https://pt.wikipedia.org/wiki/Jogo_da_velha}{\texttt{ https\+://pt.\+wikipedia.\+org/wiki/\+Jogo\+\_\+da\+\_\+velha}}
\item Lig4 -\/ \href{https://pt.wikipedia.org/wiki/Lig_4}{\texttt{ https\+://pt.\+wikipedia.\+org/wiki/\+Lig\+\_\+4}}
\item Reversi -\/ \href{https://en.wikipedia.org/wiki/Reversi}{\texttt{ https\+://en.\+wikipedia.\+org/wiki/\+Reversi}}
\end{DoxyItemize}

O programa executa\+:
\begin{DoxyEnumerate}
\item Construção de cada jogo utilizando conceitos como hierarquia, polimorfismo e composição.
\item Cadastro, armazenamento e edição de dados dos jogadores.
\item Realização de testes para verificar um código limpo, coberto e reútilizavel.
\item Interação com um oponente de Inteligência Artificial Com isso precisamos trabalhar com todos os conceitos apresentados até aqui para produzir este código.
\end{DoxyEnumerate}

Estrutura
\begin{DoxyItemize}
\item 
\begin{DoxyItemize}
\item {\bfseries{Classe jogador}} -\/ Uma classe que foi utilizada para guardar dados simples dos jogadores que será usada com composição.
\item {\bfseries{Armazenamento de dados}} -\/ Para o tratamento de dados como cadastro, remoção e modificações nas vitórias e derrotas foi utilizado um arquivo.\+txt. Isto garante o armazenamento de dados para múltiplas execuções do código. O código realiza a coleta e salvamento de dados recorrentemente para minimizar a possibilidade de perda de dados em caso de quebra de funcionamento no meio da execução.
\item {\bfseries{Classe Jogos}} -\/ Uma classe virtual que abrange os métodos que serão comuns às classes filhas
\begin{DoxyItemize}
\item {\bfseries{Classe Velha}} -\/ Classe que executa o jogo da velha e é subclasse de Jogos
\item {\bfseries{Classe Lig4}} -\/ Classe que executa o jogo da Lig4 e é subclasse de Jogos
\item {\bfseries{Classe Reversi}} -\/ Classe que executa o jogo da Reversi e é subclasse de Jogos
\end{DoxyItemize}
\item {\bfseries{Tabuleiro}} -\/ Classe que cria e printa o tabuleiro de cada jogo seguindo o devido padrão
\item {\bfseries{Minimax}} -\/ Classe que cria e executa o compotamento do oponente de Inteligência Artificial
\end{DoxyItemize}
\end{DoxyItemize}

Funcionamento
\begin{DoxyItemize}
\item O funcionamento do código se dá através de comandos do usuário. É possível cadastrar novos jogadores com a classe gerente, permitindo que o jogador insira um apelido, que deverá ser único, com uma única palavra e um nome podendo ser apenas o primeiro nome ou nome e sobrenome, sendo que devem ser utilizados apenas algarismos alfabéticos. Pode-\/se também remover jogadores a partir de seu apelido único. Também é possível printar todos os jogadores cadastrados com seus respectivos dados. Indo além desses processos básicos, partimos para execução dos jogos. Após escolher o jogo desejado sua respectiva classe inicia o jogo e realizará todos os processos esperados para os jogos, estes são\+: realizar uma jogada, com base em cordenadas do tabuleiro; verificar se tal jogada é válida; verificar seu o jogador venceu após sua rodada; verificar empate; e por fim, atualizar os dados de vitória e derrota dos jogadores. ~\newline
 É possível também realizar um duelo contra a Inteligência Artificial, que segue o mesmo padrão de regras descrito anteriormente
\end{DoxyItemize}

Problemas
\begin{DoxyItemize}
\item Entre os problemas encontrados, o mais comum se deu em se tratar com entradas inesperadas. Por exemplo, uma entrada de algarismos alfabéticos quando se espera numeros inteiros. Outro problema inesperado foi lidar com o funcionameto do Git, em se tratando de que alguns dos desenvolvedores utilizavam Windows e outros Linux. Existiram outros pequenos problemas como a configuração do doctest e do doxygen, entretanto eles foram menos impactantes.
\end{DoxyItemize}

Extras
\begin{DoxyItemize}
\item Por fim, a adição extra ao código está na previamente citada Inteligência Artificial através do Minimax. Permitindo que um usuário possa aproveitar o código mesmo se não houver um oponente humano para enfrentar. 
\end{DoxyItemize}